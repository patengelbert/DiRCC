
\newglossaryentry{api}
{
    name={API},
    description={An \glsentryfirst{api} is a particular set
            of rules and specifications that a software program can follow to access and make use of the services and resources provided by another particular software program that implements that API},
    first={\glsentrylong{api}} (\glsentryname{api}),
    long={Application Programming Interface}
}

\newglossaryentry{poets}
{
	name={POETS},
    description={\glsentryfirst{poets} is a system based around many simple computation cores running at the same time that communicate with simple messages},
    first={\glsentrylong{poets}} (\glsentryname{poets}),
    long={Partial Ordered Event Triggered Systems}
}

\newglossaryentry{software node}
{
	name={software node},
    description={A software node is a single point in the overall \gls{poets} graph. It consists of a single receive, send and compute handler and acts as a destination and/or source for messages. A \gls{poets} system is composed of multiple nodes although multiple nodes can run on a single hardware instance such a \gls{cpu}}
}

\newglossaryentry{network node}
{
	name={network node},
    description={A network node is a physical entity that is part of a \todo{finish}}
}

\newglossaryentry{nios}
{
	name={NIOS},
    description={NIOS is a \gls{soft processor} architecture used by Altera to allow for a \gls{cpu} to be implemented an \gls{fpga}. It is a 16-bit architecture with low hardware and power costs}
}

\newglossaryentry{cpu}
{
	name={CPU},
    description={A \glsentryfirst{cpu} is the electronic circuitry within a computer that carries out the instructions of a computer program by performing the basic arithmetic, logical, control and input/output (I/O) operations specified by the instructions},
    first={\glsentrylong{cpu}} (\glsentryname{cpu}),
    long={Central Processing Unit}
}

\newglossaryentry{gpu}
{
	name={GPU},
    description={A \glsentryfirst{gpu} \todo{finish gpu description}},
    first={\glsentrylong{gpu}} (\glsentryname{gpu}),
    long={Graphics Processing Unit}
}

\newglossaryentry{fpga}
{
	name={FPGA},
    long={Field-Programmable Gate Array},
    first={\glsentrylong{fpga}} (\glsentryname{fpga}),
    description={A \glsentryfirst{fpga} is an integrated circuit designed to be configured by a customer or a designer after manufacturing}
}

\newglossaryentry{soft processor}
{
	name={soft microprocessor},
    description={A soft microprocessor (also called softcore microprocessor or a soft processor) is a microprocessor core that can be wholly implemented using logic synthesis}
}

\newglossaryentry{ide}
{
	name={IDE},
    long={Integrated Development Environment},
    description={An \glsentryfirst{ide} is a software application that provides comprehensive facilities to computer programmers for software development. An IDE normally consists of a source code editor, build automation tools and a debugger},
    first={\glsentrylong{ide}} (\glsentryname{ide})
}

\newglossaryentry{hls}
{
	name={HLS},
    long={High Level Synthesis},
    description={\glsentryfirst{hls}, is an automated design process that interprets an algorithmic description of a desired behaviour and creates digital hardware that implements that behaviour},
    first={\glsentrylong{hls}} (\glsentryname{hls})
}

\newglossaryentry{avalon}
{
	name={Avalon},
    description={\glsentryname{avalon} is an interface created by Altera for easy creation of complex digital systems. It supports high-speed data streaming as well as address based data access from memory or peripherals}
}

\newglossaryentry{thread}
{
	name={thread},
    description={A \glsentryname{thread} is a \todo{finish}}
}

\newglossaryentry{hal}
{
	name={HAL},
    long={Hardware Abstraction Layer},
    first={\glsentrylong{hal}} (\glsentryname{hal}),
    description={Nios II Hardware Abstraction Layer (HAL) is a well defined software layer that forms a clear distinction between application and device-level software. The HAL also provides services such as file descriptors, I/O control, and buffering, which are required by the ANSI C library functions\cite{wiki:hal}}
}

\newglossaryentry{os}
{
	name={OS},
    long={Operating System},
    first={\glsentrylong{os}} (\glsentryname{os}),
    description={An \glsentryfirst{os} is system software that manages computer hardware and software resources and provides common services for computer programs. All computer programs, excluding firmware, require an operating system to function}
}

\newglossaryentry{application}
{
	name={application},
    description={\todo{Make description}}
}

\newglossaryentry{bit-exact}
{
	name={bit-exact},
    description={\glsentryname{bit-exact} means that for two given pieces of data, when converted to a binary format are identical. Each bit of the first piece of data is exactly the same as the corresponding bit in the second piece of data}
}

\newglossaryentry{logic element}
{
	name={logic element},
    description={\todo{describe}}
}

\newglossaryentry{port}
{
	name={port},
    description={\todo{description}}
}

\newglossaryentry{edge}
{
	name={channel},
    description={\todo{describe}}
}

\newglossaryentry{source port}
{
	name={source port},
    description={\todo{describe}}
}

\newglossaryentry{destination port}
{
	name={destination port},
    description={\todo{describe}}
}

\newglossaryentry{message}
{
	name={message},
    description={\todo{describe}}
}

\newglossaryentry{tcp}
{
	name={TCP},
    long={Transmission Control Protocol},
    first={\glsentrylong{tcp}} (\glsentryname{tcp}),
    description={\glsentryname{tcp} is a standard that defines how to establish and maintain a network conversation via which application programs can exchange data.It  enables two hosts to establish a connection and exchange streams of data. \glsentryname{tcp} guarantees delivery of data and also guarantees that packets will be delivered in the same order in which they were sent}
}

\newglossaryentry{spinnaker}
{
	name={SpiNNaker},
    long={Spiking Neural Network Architecture},
    first={\glsentrylong{spinnaker}} (\glsentryname{spinnaker}),
    description={\todo{describe}}
}

\newglossaryentry{gals}
{
	name={GALS},
    long={Globally Asynchronous, Locally Synchronous},
    first={\glsentrylong{gals}} (\glsentryname{gals}),
    description={\todo{describe}}
}

\newglossaryentry{sdram}
{
	name={SDRAM},
    description={\todo{describe}}
}

\newglossaryentry{soc}
{
	name={SoC},
    long={System on a Chip},
    description={\todo{describe}},
    first={\glsentrylong{soc}} (\glsentryname{soc})
}

\newglossaryentry{fifo}
{
	name={FIFO},
    long={First In, First Out},
    first={\glsentrylong{fifo}} (\glsentryname{fifo}),
    description={A \glsentryfirst{fifo} buffer or queue is a type of buffer that uses the First In, First Out procedure. When data is added to it, it is put on the back of the existing data and when data is removed, the data comes from the front of the stored data}
}

\newglossaryentry{quartus}
{
	name={Quartus},
    description={\todo[inline]{Write description}}
}

\newglossaryentry{noc}
{
	name={NoC},
    long={Network on a Chip},
    first={\glsentrylong{noc}} (\glsentryname{noc}),
    description ={\todo{Describe}}
}

\newglossaryentry{dircc}
{
	name={DiRCC},
    long={Distributed Reconfigurable Cloud Computing},
    first={\glsentrylong{dircc}} (\glsentryname{dircc}),
    description={\todo{describe}}
}